%!TEX root = ../main.tex

\chapter{Ethical issues}
\label{cha:ethics}

As Gillian's nature is to verify a program's correctness, a natural consequence of Gillian is the discovery of bugs in applications, particularly memory-related bugs.
This was evident when Gillian was used to verify parts of AWS's encryption SDK in both C and JavaScript, which led to the discovery of 5 total bugs, two of which were noteworthy security vulnerabilities (one in C, one in JS)\cite{gillian-part2}.
This is often the case with memory-related bugs; Microsoft has reported that across their products, over the course of 2006 to 2018, consistently around 70\% of addressed security vulnerabilities concerned memory safety\cite{microsoft-memory-bugs}.
Thus, there exists the possibility of malicious parties using Gillian to uncover security vulnerabilities by finding bugs in an organisation's code and its management of memory.

However, such an eventuality would be highly unlikely, as the malicious party would have to:
\begin{itemize}
  \item Have a working understanding of separation logic
  \item Deduce the intentions of the victim's code to the extent that they can write valid and relevant assertions for said code
  \item Be lucky enough to analyse a section of code that actually contains a vulnerability
\end{itemize}

In such a constellation of events, the completion of this project might aid the attacker in understanding the nature of the discovered vulnerability.

Because of these factors, it is clear that the benefits of Gillian and the improvement of its debugging capabilities far outweigh the slight possibility in it aiding cyberattacks.
The chance of a malicious party being aided in their attacks by Gillian is negligible compared to the possibility of the potential victim using Gillian to discover and subsequently fix vulnerabilities before they are discovered.
