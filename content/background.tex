\chapter{Background}
\label{sec:background}

\section{The path to Separation Logic}
\todo{Pinch from previous reports}
\subsection{Hoare Logic}

\subsection{Separation Logic}

\section{OCaml}

Created in 1996, OCaml~\cite{ocaml} is `an industrial-strength programming
language supporting functional, imperative and object-oriented styles'.
Specifically, OCaml provides the ability to make use of functional, imperative,
and object-oriented styles. This, combined with its powerful type system,
makes it a good fit for symbolic analysis tools. Many of Gillian's
contemporaries, including Infer~\cite{infer},
Verifast~\cite{verifast-paper, verifast-repo}, and Frama-C~\cite{frama-c}, are
also written in OCaml.

OCaml's most popular build system is Dune~\cite{dune} - it allows the easy
compilation of multi-file OCaml programs and libraries, including dependencies
from OPAM~\cite{opam}, OCaml's designated public library repository. Despite
Dune's popularity, it is a fairly barebones system, quickly growing cumbersome
when developing larger projects and, in particular, switching development
between projects; Dune requires that OPAM dependencies are installed
system-wide, which can lead to inconsistent builds and version conflicts. The
answer to these concerns is esy~\cite{esy}, a package management system for
OCaml and Reason styled after npm~\cite{npm}. Esy provides `provides a fast and
powerful workflow for local development of opam packages without requiring
"switches"', meaning dependencies are isolated between projects. Esy also
provides a number of convenience features, such as defining custom commands
that run within the OCaml project's environment.


%% Described in Intro
%\section{Gillian}
%\section{Debugging - previous work}

\section{Related work}
\label{sec:background:related-work}

\subsection{Infer}

\subsection{Verifast}

\subsection{Frama-C}


\section{VSCode and its extensions}

VSCode (Visual Studio Code)~\cite{vscode} is a universally recognised text
editor and development environment, developed by Microsoft. It currently stands
as the most popular development worldwide, enjoying a 71\% market share
according to Stack Overflow's 2021 developer survey
~\cite{stack-overflow-survey-editors}.

A large factor to VSCode's success is its extensive support for extensions,
boasting over 24,000 freely-distributed extensions~\cite{vscode-popularity},
ranging from language-specific support to general developer creature comforts
(such as SSH support). Developing new extensions is made very accesible through
simple JavaScript or TypeScript projects~\cite{vscode-extensions-intro}, making
use of VSCode's well-documented API~\cite{vscode-api} to have a large degree of
control over the editor.

Whilst VSCode limits the extent to which extensions can extend the editor's UI,
extension authors are able to make use of Webviews~\cite{vscode-webview}; an
extension can open an editor tab that contains web content (HTML/CSS/JS) that
the extension has complete control over, providing a method for extensions to
fill in any gaps in VSCode's UI.

\missingfigure{VSCode Webview example}

\subsection{Debug Adapter Protocol}
A common issue when adding IDE support for development tools is the sheer
number of editors that would ideally be supported~\cite{magpiebridge}. Whilst
VSCode has a huge market share, only supporting VSCode would neglect a
significant portion of developers. Unfortunately, maintaining support for many
editors at once is an unreasonably large undertaking. The \textit{Debug Adapter
Protocol} (DAP)~\cite{dap} exists as an attempted countermeasure to this issue,
turning the $O(m \times n)$ complexity problem of supporting many debuggers on
many editors into a $O(m + n)$ problem; each debugger and each editor need only
be integrated once.

\missingfigure{DAP integration diagram}

As discussed in \autoref{sec:intro:debugging}, the DAP's limited set of
included commands proved a limiting factor in the development of a Gillian
debugger. A potential solution is to make use of custom commands and events,
which is supported both by VSCode (sending custom requests at~\cite{vscode-dap-custom-request}, receiving custom events at~\cite{vscode-dap-custom-event} and the OCaml DAP
implementation currently used in Gillan's debugger~\cite{ocaml-dap} (where
custom events and commands are defined similarly to those already
provided~\cite{ocaml-dap-custom}).


\subsection{Including OCaml code}

Resulting of Ocsigen's work as part of their OCaml web framework
~\cite{ocsigen-framework}, OCaml programs can be directly compiled into
JavaScript using their \texttt{Js\_of\_ocaml} package~\cite{js-of-ocaml}; by
adding a compilation flag in a project's Dune file, the relevant program can be
compiled to a \texttt{.js} file instead of a native binary. This produced file
doesn't need to be used as its own program; using the provided JS
bindings~\cite{js-of-ocaml-bindings}, the resulting JS code can export
functions and values, just like normal JS code, to be used elsewhere in a
JavaScript project.

Another option is to make use of the OCaml VSCode bindings
~\cite{vscode-ocaml-bindings} provided by OCamlLabs'
\texttt{vscode-ocaml-platform}~\cite{vscode-ocaml-platform, ocamllabs}; this
way, no JavaScript need be written at all.

These options allow the line between JavaScript and OCaml to be drawn wherever
is most opportune; the extension could be written completely in JavaScript,
requiring more attention when interacting with Gillian's debugger; the
extension could be entirely OCaml, sacrificing more succinct integration with
VSCode for a more unified codebase with Gillian; it is just as possible to draw
the line somewhere inbetween, exposing the more Gillian-related OCaml code via
functions to the JavaScript that communicates with VSCode. Additionally, both
scenarious allow for Gillian code to be shared with the extension - however,
with the Gillian libraries as they are, such libraries cannot be haphazardly
included in the extension code; \texttt{Js\_of\_ocaml} transitively includes all
OCaml dependencies, potentially leading to code that cannot be executed. As a
concrete example to this, attempting to include Gillian's \texttt{debugAdapter}
library results in erroring code, due to the compiler's attempt to include an
SQLite implimentation, which it simply is not equipped to handle. Aside from
this, the inclusion of unnecessary dependencies in JS-compiled code results in
a hugely bloated extension - the \texttt{debugAdapter} example created a
compiled JS file over a million lines long. The ideal scenario here is to move
shared Gillian code to its own library within Gillian, which has as few
dependencies as possible (ideally none), but both Gillian and the extension
code depends on.

\subsection{Putting it all together}

A combination of DAP custom events and VSCode's Webviews can allow us to move
past DAP's initial limitations, however this comes at the cost of creating an
implementation specific to VSCode, eliminating a large benefit of using the DAP
in the first place.

A working example that tests custom DAP commands/events, Webviews and the
inclusion of OCaml code (alongside a Gillian library) is available at
\cite{debugger-experiment}.

\section{MagpieBridge}
