%!TEX root = ../main.tex

\chapter{Project plan}
\label{cha:plan}

A rough timeline of the project is as follows:

\newcommand{\vertline}{\color{black}\makebox[0pt]{\textbullet}\hskip-0.5pt\vrule width 1pt\hspace{\labelsep}}

\begin{flushleft}
\begin{tabular}{@{\,}r <{\hskip 2pt} !{\vertline} >{\raggedright\arraybackslash}p{13cm}}

  December 2021 & - Familiarise with previous project work           \newline
                  - Begin preliminary research on VSCode and its API,
                  and the DAP                                        \newline \\
  January 2022  & $\longleftarrow$ \textit{Current time}             \newline
                  - Read up on VSCode extensions, webviews, and the
                  compilation of OCaml code to JavaScript            \newline
                  - Complete an initial experiment with these
                  findings                                           \newline
                  - Compile research, complete interim report        \newline \\
  February & - Deep dive into Gillian debugging \& unification
             codebase                                                \newline \\
  March    & - Implement stepping through unification plans          \newline \\
  April    & - Implement custom debugging UI for VSCode, including
             branching execution and tree visualisation              \newline
             - Add error lifting support for JavaScript debugging    \newline \\
  May      & - Implement quality-of-life changes                     \newline
             - Implement Rust debugging (if possible)                \newline \\
  June     & - Finish implementation, finalize report                \newline \\

\end{tabular}
\end{flushleft}

Some extra details for parts of the plan:

\begin{itemize}
  \item The research concerning VSCode and its webviews and extensions, the
        DAP, and OCaml to JS compilation has already been completed; this
        serves the basis for \autoref{sec:background:engineering-tools}.

  \item \textbf{Unification}. A desired feature is the ability to step through
        Gillian's unification process, which stands as one of the pain points
        of the current user experience; a large share of verification errors
        occur when unifying against either the pre-condition of a called
        function, or the post-condition of the current function. In the
        presence of several specifications, the process of identifying which
        specs \textit{should} be unifying (but aren't) often takes a
        significant amount of time and thought, when a bespoke interface would
        greatly streamline this process.

        In particular, unification is performed following tree-shaped
        \textit{unification plans}; the ability to step through and visualise
        progress in these plans would immediately give the developer a far
        better idea of a unification error's cause.

  \item The `quality-of-life' features penned in for later in the project
        represent small tweaks or features that the Gillian team (or any
        testers) think would be useful additions, provided there is ample time
        to implement them.

  \item Implementing a Rust debugger depends on the progress made in another
        final year project running parallel to this one; its inclusion is not
        guaranteed, but it is also not necessary to the success of this project.

\end{itemize}
