%!TEX root = ../main.tex

\chapter{Project plan}
\label{cha:plan}

The goals of the project, together with a rough timeline, are as follows:

\newcommand{\vertline}{\color{black}\makebox[0pt]{\textbullet}\hskip-0.5pt\vrule width 1pt\hspace{\labelsep}}

\begin{flushleft}
\begin{tabular}{@{\,}r <{\hskip 2pt} !{\vertline} >{\raggedright\arraybackslash}p{13cm}}

  December 2021 & Familiarise with previous project work           \newline
                  Begin preliminary research on VSCode and its API,
                  as well as the DAP                                 \newline \\
  January 2022  & $\longleftarrow$ \textit{Current time}             \newline
                  Read up on VSCode extensions, webviews, and
                  OCaml-to-JavaScript compilation            \newline
                  Complete an initial experiment with these
                  findings                                           \newline
                  Compile background research, complete interim report        \newline \\
  February & Deep dive into Gillian codebase, with focus on 
             debugging and unification                                                \newline \\
  March    & Implement stepping through unification plans          \newline \\
  April    & Implement custom debugging UI for VSCode, including
             branching execution and tree visualisation              \newline
             Add error lifting support for JavaScript debugging    \newline \\
  May      & Implement quality-of-life changes                     \newline
             Implement Rust debugging (if possible)                \newline \\
  June     & Finish implementation, finalise report                 
\end{tabular}
\end{flushleft}

Further, additional details concerning some parts of the plan are as follows:

\begin{itemize}
  \item The research concerning VSCode and its webviews and extensions, the
        DAP, and OCaml to JS compilation has already been completed; this
        serves the basis for \autoref{sec:background:engineering-tools}.

  \item \textbf{Unification}. A desired feature is the ability to step through
        Gillian's unification process, which stands as one of the pain points
        of the current user experience; a large share of verification errors
        occur when unifying against either the pre-condition of a called
        function, the post-condition of the current function, or the folding of a predicate. In the
        presence of several specifications, the process of identifying which
        specification or predicate definition \textit{should} be unifying (but isn't) often takes a
        significant amount of time and thought, when a bespoke interface would
        greatly streamline this process.

        In particular, unification is performed following tree-shaped
        \textit{unification plans}; the ability to step through and visualise
        progress in these plans would immediately give the developer a far
        better idea of the cause of a given unification error.

  \item The `quality-of-life' features penned in for later in the project
        represent small tweaks or features that the Gillian team (or any
        testers) think would be useful additions, provided there is ample time
        to implement them.

  \item Implementing a Rust debugger depends on the progress made in another
        final-year project running parallel to this one; its inclusion is not
        guaranteed, but it is also not necessary to the success of this project.

\end{itemize}

\section{Evaluation Plan} 

The measure of success for this project comes down to three factors:

\myparagraph{Evaluating against tools}
As it stands, VeriFast is Gillian's closest competitor; once the project is
completed, Gillian's debugger will be re-compared against VeriFast to see which
has the more usable debugging experience, and why. Another comparison worth
considering is Infer; given its widespread use in industry, the pursuit of
pushing Gillian's ease of use and clarity of error reporting to be comparible
to that of Facebook's tool is a worthy goal.

\myparagraph{Evaluating against examples}
This is a more personal form of evaluation; before embarking on this project,
the author of this project sat the course on Scalable Software Verification (a.k.a.~Separation Logic); common challenges set in this course were to create lemmas,
pre-/post-conditions, loop invariants, etc. that are appropriate for particular
programs, and to verify their correctness. A good measure of Gillian's progress
would be to attempt to use it to aid with solving such challenges. 
This part of the evaluation would ideally involve usability feedback from the members of the Verified Software group, as well as some of the students who sat the Separation Logic course.

\myparagraph{Pushing the debugger's limits}
The final frontier, as it were---upon completion of this project's
implementation, the debugger ought to be pushed to its limits, exploring the
`corners' of the debugger and what it can accomplish. This will be both a good
test of the debugger's capabilities, and a good indication of what future work
may be needed.

