%!TEX root = ../../main.tex
%chktex-file 18

\section{Rethinking Gillian's Internal Mechanisms}

\subsection{Making the Interpreter Branch-Aware}%
\label{sec:interpreter-branching}

A particularly interesting consequence of implementing the given logging
structure was the inclusion of \texttt{branch\_case} fields, both in
\texttt{"cmd"} and \texttt{"cmd\_result"} reports (see \autoref{lst:branch-case}
for the relevant type). As can be seen in \autoref{fig:log-structure-desired},
the multiple results of a command that branches (such as an if/else), and the
relevant commands that directly follow, are associated with eachother via their
\texttt{branch\_case} fields.

\texttt{branch\_case}s could then be used as a mechanism for dictating branched
execution. Each state configuration type has been extended with a field of type
\texttt{branch\_path} (equivalent to a \texttt{branch\_case list}),  denoting
the `path' of branch cases that had been taken to reach this point in symbolic
execution. A branch path can optionally be passed to the continuation function,
which would then be used to select the next state configuration. Passing no
branch path preserves original behaviour, popping the first available state
config.

This implementation required a change to the \texttt{cont\_func} type for
handling the ends of branches, as the end of a branch requires result data for
use in unifications, but the provided \texttt{Finished} variant gives no
continuation function. A new variant, \texttt{EndOfBranch}, shown in
\autoref{lst:contfunc-type-new} has been introduced to cover the case where
there may still be state configurations left to consider, but none that match
the provided branch path. This variant gives both the results for the relevant
symbolic execution branch, and a cont func that can be used to explore other
branches. The \texttt{Normal} variant has been extended to include information
on the branch path, and any new branch cases resulting from the execution step.

\begin{lstlisting}[caption={
  The new \texttt{cont\_func} type, in the \texttt{GInterpreter} module
  \label{lst:contfunc-type-new}}, style=code, numbers=none]
type 'a cont_func_f = ?path:branch_path -> unit -> 'a cont_func

and 'a cont_func =
  | Finished of 'a list
  | Continue of
      (Logging.ReportId.t option
      * branch_path
      * branch_case list option
      * 'a cont_func_f)
  | EndOfBranch of 'a * 'a cont_func_f
\end{lstlisting}


\subsection{Navigating Symbolic Execution}

Despite a mechanism now being in place for selecing the branch of execution, it
won't be of much use to users without a means of navigating the various paths
through a symbolic execution. To this end, an `execution map', or
\texttt{ExecMap} (\autoref{lst:execmap}) was developed to keep track of that
very process.

\begin{lstlisting}[caption={
  The \texttt{ExecMap} module, inside the \texttt{Debugger} module
  \label{lst:execmap}}, style=code, numbers=none]
module ExecMap = struct
  type unifys = (rid * Unifier.unify_kind * UnifyMap.unify_result) list
  [@@deriving yojson]

  type cmd_data = {
    id : rid;
    origin_id : int option; [@name "originId"]
    display : string;
    unifys : unifys;
    errors : string list;
  }
  [@@deriving yojson]

  type 'case t =
    | Nothing
    | Cmd of cmd_data * 'case t
    | BranchCmd of cmd_data * ('case * 'case t) list
    | FinalCmd of cmd_data
  [@@deriving yojson]

  (* ... helper functions ... *)

end
\end{lstlisting}

After each call of the continuation function, the relevant information from
the executed step is added to the \texttt{ExecMap} in the debugger's internal
state. This map is used at various points in the debugging process, such as
finding the report ID of a command at a particular branch path (and vice versa),
and listing the different branch cases that the interpreter could step to after
a particular command.

The \texttt{ExecMap} can also be supplied wholesale to the debug adapter,
albeit with \texttt{PackagedBranchCase} instead of \texttt{branch\_case}.
The reason for this, and for the \texttt{'case} type variable in
\texttt{ExecMap}, is to abstract away the details of the \texttt{branch\_case}
type for the debugger frontend. \texttt{PackagedBranchCase}
(\autoref{lst:packagedbranchcase}) simply provides the `kind' of branch case,
a pretty-printed form of both the branch case's kind and value, and the JSON
for the \texttt{branch\_case} represented. This allows the frontend to directly
use a representation fit for the user's viewing, while still holding on to the
necessary data structure to talk back to the debugger.

\begin{lstlisting}[caption={
  The \texttt{PackagedBranchCase} type, inside the \texttt{Debugger} module
  \label{lst:packagedbranchcase}}, style=code, numbers=none]
module PackagedBranchCase = struct
  type t = {
    kind : string;
    display : string * string;
    json : Yojson.Safe.t;
  }
  [@@deriving yojson]

  (* ... helper functions ... *)

end
\end{lstlisting}

\subsection{Enabling Unification When Debugging}
\todo{Explan what was needed for unifying while debugging}
\todo{Talk about UnifyMap}
