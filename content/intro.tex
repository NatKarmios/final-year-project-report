%!TEX root = ../main.tex

\chapter{Introduction}\label{sec:intro}

As programs become larger and more complex, how do we guarantee the correctness
of code? Symbolic analysis tools are becoming increasingly promising as research
and development continues. These generally come in the form of {\bf (a)}
automated testing via bi-abduction, where code is submitted as-is for symbolic
executed, with potential errors (such as divide-by-zero and null dereferences)
reported as a result, and {\bf (b)} full program verification, where code is
annotated with logical conditions (pre-conditions, post-conditions, loop
invariants, etc.); while bi-abductive techniques are far more developer-friendly
(and thus have seen more industry use, such as with Facebook's Infer
~\cite{infer}), full verification guarantees correct behaviour rather than
merely pointing out specific kinds of bugs. Particularly in the case of full
verification, the lack of user-friendly tooling remains a prohibitive factor for
wider adoption.

Full verification tools do exist, albeit with limitations. The most accessible
tool currently available is VeriFast~\cite{verifast-paper, verifast-repo},
allowing users to verify C and Java code. VeriFast's limitations lie in its
static nature; verification is performed in one fell swoop, and the user is
only given information at a single point of execution, should an error or a
breakpoint be encountered; if one wanted to explore the symbolic execution
of the program, verification must be re-run for each step, moving the breakpoint
along between each execution. Another big player in this space --- the focus of
this project --- is Gillian, a platform for symbolic analysis parametric on the
target language and its memory model. Until recently, Gillian was only usable
by specialists who understood its internals, as the only way to debug
verification was to sift through an overly verbose log file. Some previous MEng
projects have gone some way to improve user-friendliness, though the majority of
this work was on Gillian's internals, with a user-facing debugger first being
attempted last year (albeit with limited results due to strict adherence to
Microsoft's Debug Adapter Protocol).

This project focuses on delivering a rich debugging experience, with user
accessibility as a priority. Gillian was not entirely ready for this, so some
reconfiguration of its internals was required, in particular: {\bf (a)}
restructuring Gillian's internal log database to closer represent an intuitive
understanding of the symbolic execution and unification processes
(\autoref{sec:log-structure}); {\bf (b)} overhauling Gillian's symbolic
interpreter to support switching between branches of execution at will, while
still supporting the original sequential behaviour
(\autoref{sec:interpreter-branching}); {\bf (c)} creating and populating
navigable data structures to help traverse symbolic execution and unification
(\autoref{sec:navigating-symex}, \autoref{sec:debug-unify}); {\bf (d)} creating
custom events and commands alongside those provided by the Debug Adapter
Protocol to make the debugger fit for symbolic execution
(\autoref{sec:dap-custom}). These changes allowed the desired debugging
interface to come to fruition; Visual Studio Code's built-in debugger interface
was extended with an interface to navigate both symbolic execution and
unification through the use of webviews and bidirectional communication with
the debugger.

The completion of this project leads to even bigger goals; there exists a very
real possibility of using Gillian's new debugging capabilities for coursework in
the next iteration of the separation logic course taught to undergrads at
Imperial. From there, we look to industry; with better target language lifting
and some more niceties in the interface, the future of Gillian may see the
barrier to entry for developers to employ full verification reduced to only a
working knowledge of separation logic.
