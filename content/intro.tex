%!TEX root = ../main.tex

\chapter{Introduction}
\label{sec:intro}

% Why is Gillian necessary?

% Why is *debugging* Gillian necessary?

% Why should we extend the debugging capabilities.

It is an inevitable truth that the complexity of programs and their codebase
will continuously grow. As this happens, how can one guarantee code does what
it needs to?

%\section{Symbolic execution and analysis}

Symbolic analysis tools are becoming an increasingly promising choice as
research and development into a plethora of projects continue (some of which
are discussed in \autoref{sec:background:sl-and-tools}). The benefit of using
symbolic methods over ones dependent on concrete execution (i.e. unit tests) is
that they cover all possible states of execution without needing to specify each
scenario by hand.

%\section{Gillian}

Enter
Gillian~\cite{gillian-santos,gillian-part1,gillian-part2,gillian-techrep}.
Created by Imperial College London's Verified Software group, it
is a platform for the development of symbolic analysis tools. Compared to
related projects, Gillian has two advantages.
Firstly, it is \textit{parametric} on the target language's (TL) memory model;
if provided with a compiler from a TL to GIL (a simple, Gillian-specific
GOTO language) and an implementation of the TL's memory model (in OCaml),
then Gillian can do all the heavy lifting and perform symbolic analysis on programs written in that
TL in various ways~\cite{gillian-part1}. This brings to front Gillian's second advantage; it supports three methods of symbolic
analysis:
\begin{itemize}
  \item \emph{Whole program symbolic testing}, in which developers write a
  number of unit tests with symbolic inputs and outputs, as well as constraints
  on the outputs in first-order logic. Gillian then attempts to find a symbolic
  state that invalidates those assertions.

  \item \emph{Full verification}: developers annotate functions with pre- and
  post-conditions, loop invariants, and proof tactics, in the style of
  separation logic; Gillian then symbolically executes the functions with
  pre-conditions intact to see if the post-conditions hold.

  \item \emph{Automatic compositional testing}. Similarly to Facebook's
  Infer~\cite{infer, infer-site}, developers submit a program without any
  pre-conditions, post-conditions, or any other logical annotations, and,
  through bi-abduction\cite{bi-abduction}, Gillian will respond with
  separation logic-style specifications for that programs behaviour, up
  to a specified bound.
\end{itemize}


%\section{Debugging}
%\label{sec:intro:debugging}

Whilst symbolic analysis tools have clear benefits when proving program
correctness, the adoption of symbolic methods in industry remains limited; a
strong factor to this end is the lack of IDE support for
developers~\cite{magpiebridge}.
%
In addition, in the specific case of Gillian, upon failure of any of its analyses, 
a user has to sift through a multi-megabyte log file to
find the offending piece of GIL code; deducing both the cause of the error and
the location of the problems in the original TL code would take an
unreasonable amount of work and understanding~\cite{gillian-debugging-2021}.

To assuage these issues, significant work has gone into developing a debugger
for Gillian's verification process, as well as improving its error reporting.
This undertaking, in its current state, makes use of the DAP (debug adapter protocol) 
to provide a debugger than can be dropped into any IDE that supports it, with little-to-no
integration required with any specific IDE. Whilst this debugger is a huge
improvement to the user experience of Gillian, the DAP only provides a limited
amount of functionality, primarily designed for debugging concrete execution.
The decision had been taken to work within these limitations, producing a
debugger that can successfully step over, into and out of (GIL, WISL, or
JavaScript) code. An unfortunate consequence of this decision is that a number
of much desired features could not be provided, such as `the ability to choose
which branch to go down on a conditional statement' and `better visualisations
of the current execution tree'~\cite[p.~49]{gillian-debugging-2021}.

The decision to use the DAP instead of a custom solution was made, quite
reasonably so, due to `the short timeframe of a Master's project'. Standing on
the shoulders of previous work to this end, this project aims to take Gillian's
debugging tools beyond the limitations of the DAP, and provide an experience
befitting real users, rather than experimental scenarios. 
