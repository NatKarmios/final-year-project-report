%!TEX root = ../main.tex
%chktex-file 36

\appendix
\chapter{Listings}

\noindent\rule{\textwidth}{0.5pt}
\vspace{-0.6cm}
\begin{minted}{ocaml}
(** Allowed strings for the type_ field of a report *)
module ContentType = struct
  let debug = "debug"
  let phase = "phase"
  let cmd_step = "cmd_step"
  let annotated_action = "annotated_action"
  let set_freed_info = "set_freed_info"
end
\end{minted}
\vspace{-0.4cm}
\noindent\rule{\textwidth}{0.5pt}
\vspace{-0.6cm}
\captionof{listing}{\texttt{loggingConstants.ml}, pre-project}%
\label{lst:loggingconstants}

\vspace{2em}
\noindent\rule{\textwidth}{0.5pt}
\vspace{-0.6cm}
\begin{minted}{ocaml}
type cconf_t =
| ConfErr of {
    callstack : CallStack.t;
    proc_idx : int;
    error_state : state_t;
    errors : err_t list;
  }
| ConfCont of {
    state : state_t;
    callstack : CallStack.t;
    invariant_frames : invariant_frames;
    prev_idx : int;
    next_idx : int;
    loop_ids : string list;
    branch_count : int;
  }
| ConfFinish of {
    flag : Flag.t;
    ret_val : state_vt;
    final_state : state_t
  }
    (** Equal to Conf cont + the id of the required spec *)
| ConfSusp of {
    spec_id : string;
    state : state_t;
    callstack : CallStack.t;
    invariant_frames : invariant_frames;
    prev_idx : int;
    next_idx : int;
    loop_ids : string list;
    branch_count : int;
  }
\end{minted}
\vspace{-0.4cm}
\noindent\rule{\textwidth}{0.5pt}
\vspace{-0.6cm}
\captionof{listing}{The \texttt{cconf\_t} type in \texttt{GInterpreter.ml}, pre-project}
\label{lst:cconf-type}

\vspace{2em}
\noindent\rule{\textwidth}{0.5pt}
\vspace{-0.6cm}
\begin{minted}{ocaml}
(** Module specifying functions required for a type to be loggable *)
module type S = sig
  (** Type to be logged *)
  type t [@@deriving yojson]

  (** Pretty printer for the type *)
  val pp : Format.formatter -> t -> unit
end

(* Type for a module which specifies functions required for a type to be loggable *)
type 'a unpacked = (module S with type t = 'a)

(** Type storing the functions required to log the specified type and the
    actual content to be logged *)
type t = L : ('a unpacked * 'a) -> t

(** Calls the pretty print function of a loggable on its content *)
let pp (loggable : t) (formatter : Format.formatter) =
  match loggable with
  | L (t, content) ->
      let (module T) = t in
      T.pp formatter content

(** Converts a loggable to Yojson *)
let loggable_to_yojson = function
  | L (t, content) ->
      let (module T) = t in
      T.to_yojson content

(** Returns a loggable, given the required functions and content *)
let make
    (type a)
    (pp : Format.formatter -> a -> unit)
    (of_yojson : Yojson.Safe.t -> (a, string) result)
    (to_yojson : a -> Yojson.Safe.t)
    (content : a) : t =
  let module M = struct
    type t = a

    let pp = pp
    let of_yojson = of_yojson
    let to_yojson = to_yojson
  end in
  L ((module M), content)

(** Returns a loggable given a string to be logged *)
let make_string (s : string) : t =
  let pp = Fmt.string in
  let of_yojson yojson =
    match yojson with
    | `String s -> Ok s
    | _ -> Error "Cannot parse yojson to a string"
  in
  let to_yojson s = `String s in
  make pp of_yojson to_yojson s
\end{minted}
\vspace{-0.4cm}
\noindent\rule{\textwidth}{0.5pt}
\vspace{-0.6cm}
\captionof{listing}{The \texttt{Loggable} module}
\label{lst:loggable}

\vspace{2em}
\noindent\rule{\textwidth}{0.5pt}
\vspace{-0.6cm}
\begin{minted}{ocaml}
let set_previous = ReportBuilder.set_previous
let get_parent = ReportBuilder.get_parent
let set_parent = ReportBuilder.set_parent
let release_parent = ReportBuilder.release_parent

let with_parent_id id f =
  match id with
  | None -> f ()
  | Some id -> (
      set_parent id;
      let result =
        try Ok (f ())
        with e ->
          Printf.printf "Original Backtrace:\n%s" (Printexc.get_backtrace ());
          Error e
      in
      release_parent (Some id);
      match result with
      | Ok ok -> ok
      | Error e -> raise e)

let with_parent ?title ?(lvl = Mode.Normal) ?severity loggable type_ f =
  let id =
    Option.bind loggable (fun loggable ->
        log_specific lvl ?title ?severity loggable type_)
  in
  with_parent_id id f
\end{minted}
\vspace{-0.4cm}
\noindent\rule{\textwidth}{0.5pt}
\vspace{-0.6cm}
\captionof{listing}{Excerpt from the \texttt{Logging} module, showing the new
functions for controlling the parent and previous report IDs}%
\label{lst:logging-parent-previous}

\vspace{2em}
\noindent\rule{\textwidth}{0.5pt}
\vspace{-0.6cm}
\begin{minted}{ocaml}
let previous : ReportId.t option ref = ref None
let parents : ReportId.t Stack.t = Stack.create ()

let make
    ?title
    ~(content : Loggable.t)
    ~(type_ : string)
    ?(severity = Report.Log)
    () =
  let title =
    match title with
    | None -> type_
    | Some title -> title
  in
  let report : Report.t =
    {
      id = ReportId.next ();
      title;
      elapsed_time = Sys.time ();
      previous = !previous;
      parent = Stack.top_opt parents;
      content;
      severity;
      type_;
    }
  in
  previous := Some report.id;
  report

let set_previous = function
  | None -> ()
  | Some _ as id -> previous := id

let get_parent () = Stack.top_opt parents

let set_parent id =
  previous := None;
  Stack.push id parents

let release_parent = function
  | None -> ()
  | Some rid as id_opt ->
      let parent_id = Stack.pop parents in
      assert (ReportId.equal rid parent_id);
      previous := id_opt

let start_phase level ?title ?severity () =
  if Mode.should_log level then (
    let report =
      make ?title
        ~content:(Loggable.make_string "*** Phase ***")
        ~type_:LoggingConstants.ContentType.phase ?severity ()
    in
    set_parent report.id;
    Some report)
  else None

let end_phase = release_parent
let get_cur_parent_id () = Stack.top_opt parents
\end{minted}
\vspace{-0.4cm}
\noindent\rule{\textwidth}{0.5pt}
\vspace{-0.6cm}
\captionof{listing}{The \texttt{ReportBuilder} module, including new functions for manually
  controlling the parent and previous report IDs}%
\label{lst:reportbuilder}

\vspace{2em}
\noindent\rule{\textwidth}{0.5pt}
\vspace{-0.6cm}
\begin{minted}{ocaml}
module Logging = struct
  let pp_str_list = Fmt.(brackets (list ~sep:comma string))

  module ConfigReport = struct
    type t = {
      proc_line : int;
      time : float;
      cmd : string;
      callstack : CallStack.t;
      annot : Annot.t;
      branching : int;
      state : state_t;
      branch_case : branch_case option;
    }
    [@@deriving yojson, make]

    let pp
        state_printer
        fmt
        {
          proc_line = i;
          time;
          cmd;
          callstack = cs;
          annot;
          branching;
          state;
          _;
        } =
      Fmt.pf fmt
        "@[------------------------------------------------------@\n\
         --%s: %i--@\n\
         TIME: %f@\n\
         CMD: %s@\n\
         PROCS: %a@\n\
         LOOPS: %a ++ %a@\n\
         BRANCHING: %d@\n\
         @\n\
         %a@\n\
         ------------------------------------------------------@]\n"
        (CallStack.get_cur_proc_id cs)
        i time cmd pp_str_list
        (CallStack.get_cur_procs cs)
        pp_str_list
        (Annot.get_loop_info annot)
        pp_str_list
        (CallStack.get_loop_ids cs)
        branching state_printer state

    let to_loggable state_printer =
      L.Loggable.make (pp state_printer) of_yojson to_yojson

    let log state_printer report =
      L.normal_specific
        (to_loggable state_printer report)
        L.LoggingConstants.ContentType.cmd
  end

  module CmdResult = struct
    type t = {
      callstack : CallStack.t;
      proc_body_index : int;
      state : state_t option;
      errors : err_t list;
      branch_case : branch_case option;
    }
    [@@deriving yojson]

    let pp fmt cmd_step =
      (* TODO: Cmd step should contain all things in a configuration
               print the same contents as log_configuration *)
      CallStack.pp fmt cmd_step.callstack

    let to_loggable = L.Loggable.make pp of_yojson to_yojson
    let log type_ report = L.normal_specific (to_loggable report) type_

    open L.LoggingConstants.ContentType

    let log_result = log cmd_result
    let log_init = log proc_init
    let log_step = log cmd_step
  end

  module AnnotatedAction = struct
    type t = { annot : Annot.t; action_name : string } [@@deriving yojson]

    let pp fmt annotated_action =
      let origin_loc = Annot.get_origin_loc annotated_action.annot in
      Fmt.pf fmt "Executing action '%s' at %a" annotated_action.action_name
        (Fmt.option ~none:(Fmt.any "none") Location.pp)
        origin_loc

    let to_loggable = L.Loggable.make pp of_yojson to_yojson

    let log report =
      L.normal_specific (to_loggable report)
        L.LoggingConstants.ContentType.annotated_action
  end

  let log_configuration
      (cmd : Annot.t * int Cmd.t)
      (state : State.t)
      (cs : CallStack.t)
      (i : int)
      (b_counter : int)
      (branch_case : branch_case option) : L.ReportId.t option =
    let annot, cmd = cmd in
    let state_printer =
      match !Config.pbn with
      | false -> State.pp
      | true ->
          let pvars, lvars, locs =
            (Cmd.pvars cmd, Cmd.lvars cmd, Cmd.locs cmd)
          in
          State.pp_by_need pvars lvars locs
    in
    ConfigReport.log state_printer
      (ConfigReport.make ~proc_line:i ~time:(Sys.time ())
         ~cmd:(Fmt.to_to_string Cmd.pp_indexed cmd)
         ~callstack:cs ~annot ~branching:b_counter ~state ?branch_case ())

  let print_lconfiguration (lcmd : LCmd.t) (state : State.t) : unit =
    L.normal (fun m ->
        m
          "@[------------------------------------------------------@\n\
           TIME: %f@\n\
           LCMD: %a@\n\
           @\n\
           %a@\n\
           ------------------------------------------------------@]@\n"
          (Sys.time ()) LCmd.pp lcmd State.pp state)

  let pp_err = ExecErr.pp Val.pp State.pp_err
  let pp_single_result ft res = ExecRes.pp State.pp Val.pp pp_err ft res

  (** Configuration pretty-printer *)
  let pp_result (ft : Format.formatter) (reslt : result_t list) : unit =
    let open Fmt in
    let pp_one ft (i, res) =
      pf ft "RESULT: %d.@\n%a" i pp_single_result res
    in
    (iter_bindings List.iteri pp_one) ft reslt
end
\end{minted}
\vspace{-0.4cm}
\noindent\rule{\textwidth}{0.5pt}
\vspace{-0.6cm}
\captionof{listing}{Logging types for the interpreter, inside the
\texttt{GInterpreter} module}%
\label{lst:interpreter-logging}

\vspace{2em}
\noindent\rule{\textwidth}{0.5pt}
\vspace{-0.6cm}
\begin{minted}{ocaml}
  module Logging = struct
    let pp_variants : (string * Expr.t option) Fmt.t =
      Fmt.pair ~sep:Fmt.comma Fmt.string (Fmt.option Expr.pp)

    let pp_astate fmt astate =
      let state, preds, _, variants = astate in
      Fmt.pf fmt "%a@\nPREDS:@\n%a@\nVARIANTS:@\n%a@\n" State.pp state Preds.pp
        preds
        (Fmt.hashtbl ~sep:Fmt.semi pp_variants)
        variants

    let pp_astate_by_need (pvars : SS.t) (lvars : SS.t) (locs : SS.t) fmt astate
        =
      let state, preds, _, variants = astate in
      Fmt.pf fmt "%a@\n@\nPREDS:@\n%a@\nVARIANTS:@\n%a@\n"
        (State.pp_by_need pvars lvars locs)
        state Preds.pp preds
        (Fmt.hashtbl ~sep:Fmt.semi pp_variants)
        variants

    module AstateRec = struct
      type t = { state : state_t; preds : preds_t; variants : variants_t }
      [@@deriving yojson]

      let from (state, preds, _, variants) = { state; preds; variants }

      let pp_custom pp_astate fmt { state; preds; variants } =
        pp_astate fmt (state, preds, (), variants)

      let pp = pp_custom pp_astate
    end

    module AssertionReport = struct
      type t = { step : UP.step; subst : ESubst.t; astate : AstateRec.t }
      [@@deriving yojson]

      let pp_custom pp_astate pp_subst fmt { step; subst; astate } =
        Fmt.pf fmt
          "Unify assertion: @[<h>%a@]@\nSubst:@\n%a@\n@[<v 2>STATE:@\n%a@]"
          UP.step_pp step pp_subst subst
          (AstateRec.pp_custom pp_astate)
          astate

      let to_loggable pp_astate pp_subst =
        L.Loggable.make (pp_custom pp_astate pp_subst) of_yojson to_yojson
    end

    module UnifyReport = struct
      type t = {
        astate : AstateRec.t;
        subst : ESubst.t;
        up : UP.t;
        unify_kind : unify_kind;
      }
      [@@deriving yojson]

      let pp fmt _ = Fmt.pf fmt "Unifier.unify: about to unify UP."
      let to_loggable = L.Loggable.make pp of_yojson to_yojson

      let as_parent report f =
        L.with_parent
          (Some (to_loggable report))
          L.LoggingConstants.ContentType.unify f
    end

    module UnifyCaseReport = struct
      type t = { astate : AstateRec.t; subst : st; up : UP.t }
      [@@deriving yojson]

      let to_loggable = L.Loggable.make L.dummy_pp of_yojson to_yojson

      let log report =
        L.normal_specific (to_loggable report)
          L.LoggingConstants.ContentType.unify_case
    end

    module UnifyResultReport = struct
      type remaining_state = UnifyCaseReport.t [@@deriving yojson]

      type t =
        | Success of {
            astate : AstateRec.t;
            subst : st;
            posts : (Flag.t * Asrt.t list) option;
            remaining_states : remaining_state list;
          }
        | Failure of {
            cur_step : UP.step option;
            subst : st;
            astate : AstateRec.t;
            errors : err_t list;
          }
      [@@deriving yojson]

      let pp fmt report =
        match report with
        | Success data ->
            Fmt.pf fmt
              "Unifier.unify_up: Unification successful: %d states left"
              (1 + List.length data.remaining_states)
        | Failure { cur_step; subst; astate; errors } ->
            Fmt.pf fmt
              "@[<v 2>WARNING: Unify Assertion Failed: @[<h>%a@] with subst @\n\
               %a in state @\n\
               %a with errors:@\n\
               %a@]"
              Fmt.(option ~none:(any "no assertion - phantom node") UP.step_pp)
              cur_step ESubst.pp subst AstateRec.pp astate
              Fmt.(list ~sep:(any "@\n") State.pp_err)
              errors

      let to_loggable = L.Loggable.make pp of_yojson to_yojson

      let log report =
        L.normal_specific (to_loggable report)
          L.LoggingConstants.ContentType.unify_result
    end
  end
\end{minted}
\vspace{-0.4cm}
\noindent\rule{\textwidth}{0.5pt}
\vspace{-0.6cm}
\captionof{listing}{Logging types for unification, inside the \texttt{Unifier} 
module}%
\label{lst:unifier-logging}

\vspace{2em}
\noindent\rule{\textwidth}{0.5pt}
\vspace{-0.6cm}
\begin{minted}{ocaml}
type branch_case =
  | GuardedGoto of bool
  | LCmd of int
  | SpecExec of Flag.t
  | LAction of state_vt list
  | LActionFail of int
[@@deriving yojson]
\end{minted}
\vspace{-0.4cm}
\noindent\rule{\textwidth}{0.5pt}
\vspace{-0.6cm}
\captionof{listing}{The \texttt{branch\_case} type}%
\label{lst:branch-case}

\vspace{2em}
\noindent\rule{\textwidth}{0.5pt}
\vspace{-0.6cm}
\begin{minted}{js}
export type TransformResult<M, D, A> = {
  id: string | number;
  data: D;
  nexts: [A, M][];
  edgeLabel?: React.ReactNode;
};

export type TransformFunc<M, D, A> = (
  map: M,
  parent: string,
  aux: A
) => TransformResult<M, D, A>;

export type Props<M, D, A> = {
  initElem: TransformResult<M, D, A>;
  transform: TransformFunc<M, D, A>;
  nodeComponent: React.ComponentType<NodeProps<D>>;
};
\end{minted}
\vspace{-0.4cm}
\noindent\rule{\textwidth}{0.5pt}
\vspace{-0.6cm}
\captionof{listing}{The required types for the \texttt{<TreeMapView />}
component}%
\label{lst:treemapview-types}

\vspace{2em}
\noindent\rule{\textwidth}{0.5pt}
\vspace{-0.6cm}
\begin{minted}{js}
// #region ExecMap

export type BranchCase = {
  readonly kind: string;
  readonly display: [string, string];
  readonly json: any;
};

export type CmdData = {
  readonly id: number;
  readonly originId: number | null;
  readonly display: string;
  readonly unifys: readonly (readonly [number, UnifyKind, UnifyResult])[];
  readonly errors: readonly string[];
};

export type ExecMap =
  | readonly ['Nothing']
  | readonly ['Cmd', CmdData, ExecMap]
  | readonly ['BranchCmd', CmdData, [BranchCase, ExecMap][]]
  | readonly ['FinalCmd', CmdData];

// #endregion

// #region UnifyMap

export type UnifyResult = readonly ['Success' | 'Failure'];

export type AssertionData = {
  readonly id: number;
  readonly fold: readonly [number, UnifyResult] | null;
  readonly assertion: string;
  readonly substitutions: readonly [string, string][];
};

export type UnifySeg =
  | readonly ['Assertion', AssertionData, UnifySeg]
  | readonly ['UnifyResult', number, UnifyResult];

export type UnifyKind = [
  'Postcondition' | 'Fold' | 'FunctionCall' | 'Invariant' | 'LogicCommand'
];

export type UnifyMap = readonly [
  UnifyKind,
  readonly ['Direct', UnifySeg] | readonly ['Fold', UnifySeg[]]
];

// #endregion

export type DebugState = {
  readonly execMap: ExecMap;
  readonly liftedExecMap: ExecMap | null;
  readonly currentCmdId: number;
  readonly unifys: readonly (readonly [number, UnifyKind, UnifyResult])[];
  readonly procName: string;
};

export type UnifyStep =
  | readonly ['Assertion', AssertionData]
  | readonly ['Result', number, UnifyResult];

export type Unification = {
  readonly map: unknown; // TODO: fix when Immer supports recursive types
  readonly selected?: UnifyStep;
};

export type UnifyState = {
  readonly path: number[];
  readonly unifications: Record<number, Unification | undefined>;
};

export type State = {
  readonly debugState?: DebugState;
  readonly unifyState: UnifyState;
};

// #region MessageToWebview

type StateUpdateMsg = {
  readonly type: 'state_update';
  readonly state: DebugState;
};

type UnifyUpdateMsg = {
  readonly type: 'unify_update';
  readonly unifyId: number;
  readonly unifyMap: UnifyMap;
};

export type MessageToWebview = StateUpdateMsg | UnifyUpdateMsg;

// #endregion

// #region MessageFromWebiew

type RequestStateUpdateMsg = {
  readonly type: 'request_state_update';
};

type RequestJump = {
  readonly type: 'request_jump';
  readonly cmdId: number;
};

type RequestExecSpecific = {
  readonly type: 'request_exec_specific';
  readonly prevId: number;
  readonly branchCase: BranchCase | null;
};

type RequestUnification = {
  readonly type: 'request_unification';
  readonly id: number;
};

export type MessageFromWebview =
  | RequestStateUpdateMsg
  | RequestJump
  | RequestExecSpecific
  | RequestUnification;

// #endregion
\end{minted}
\vspace{-0.4cm}
\noindent\rule{\textwidth}{0.5pt}
\vspace{-0.6cm}
\captionof{listing}{The types provided in \texttt{types.ts}}%
\label{lst:types-ts}