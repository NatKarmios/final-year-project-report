%!TEX root = ../main.tex

\chapter{Conclusion and Next Steps}\label{sec:conclusion}

We believe that this project constitutes a solid leap forward in Gillian's
debugging capabilities, and as a result, its overall usability to the those in
the field of separation logic and symbolic execution; in fact, existing users of
Gillian have stated their intentions to use the new UI as a means of explaining
how the underlying CSE (compositional symbolic execution) works when authoring
papers in the future. Various technical decisions underpinning Gillian's
implementation have been called into question and reimagined, such as the
logging structure and the mechanisms of the symbolic execution interpreter, and
VSCode's extensible capabilities have been incorporated to unlock Gillian's
potential.

The work completed opens a number of avenues for further research and
development, some of which are already being considered.

\myparagraph{A more ambitious tool.}
The Gillian debugger has already proven to be a great tool to visualise and
understand events occurring during symbolic execution. Regardless, there is
still much potential to be unlocked.
The first point of order is to overcome some/any/all limitations described in
\autoref{sec:eval:limitations}. Additional work can be done to push the limits
of what can be visualised, by adding navigable heap diagrams with abstractions
(i.e.\ recursive predicates). Finally, \autoref{sec:navigating-symex} shows that
user interaction could be fed back into the symbolic execution engine through
the debugger; this possibility warrants exploration, in order to make execution
more interactive and allow the user to leverage the debugger to add tactics and
annotations, automatically hot-fixing code on the fly. 

\myparagraph{Education.}
Looking forwards, work has already been planned to add what little more is
needed for Gillian to be used for educational purposes, i.e.\ in the Scalable
Software Verification course at Imperial, particularly for coursework use; the
ability to step though each command of a symbolic execution could strengthen the
students' understanding of separation logic and symbolic execution. This will
not only serve as proof of Gillian's usefulness, but also create a strong
feedback loop, as Gillian will be opened to the student body---a more diverse
group than the collection of researchers that make up the Verified Software
group---leading to a more rapid and better motivated development in Gillian's
future.

\myparagraph{Real-world development.}
The advancements made to the debugger in this project bring Gillian closer
to the prospect of industry use. Infer has proven that separation logic analysis
is of merit to the software industry, though the feasibility of a more complete
analysis---requiring developer knowledge of separation logic---has not yet
been explored. Exploration of real-world Gillian usage would make the most sense
in more security-conscious contexts; this was a large motivating factor in
Gillian's application to a section of Amazon's AWS encryption SDK.\ Where it
stands now, it is not infeasible for Gillian to bring further impact in this
context.

\myparagraph{A Symbolic DAP.}
After some consideration, it was realised that the usefulness of the extensions
made to the Debug Adapter Protocol might not be exclusive to Gillian; there may
be some possibility for the formalisation of a \textit{Symbolic} Debug Adapter
Protocol, or SDAP. In fact, initial research found that this is not the first
time such a thing has been considered, though an attempt at formalizing this
concept has yet to be attempted. Some publications to this end are limited in
scope from being masters' theses~\cite{sdap-aurecchia}, while others only
achieve support for debugging a single symbolic tool in a single
IDE~\cite{sdap-colombo, sdap-kps} (which, in fairness, has been the end result
of this project as well).
There is only one fully-realised `platform-agnostic' tool (though the API is
Java-based) aimed at improving the user experience of static analysis tools---%
MagpieBridge, as introduced in \autoref{sec:background:magpiebridge}, However,
this only serves to extend the Language Server Protocol rather than have any
debugging functionality or any interactivity whatsoever. The appropriateness of
the graphical interface developed in this project must also be considered, as it
may be worth including its functionality in the specification for an SDAP
standard.
