%!TEX root = ../main.tex

\chapter{Conclusion and Next Steps}\label{sec:conclusion}

This project serves as a monumental leap forward in Gillian's debugging
capabilities, and as a result, its overall usability to the those in the field
of separation logic \sacha{I don't know where exactly, but you can mention that your tool can be used to explain how CSE (compositional symbolic execution) works in our future papers}. Various technical decisions underpinning Gillian's
implementation have been called into question and reimagined, such as the
logging structure and the mechanisms of the symbolic execution interpreter, and
VSCode's extensible capabilities have been incorporated to unlock Gillian's
potential.

There is some difficulty in describing this project's merit, stemming somewhat
from its novelty, being the crescendo of many years of work to make automatic
program verification accessible to any who understand separation logic, not just
the authors of Gillian. The technical hurdles faced during actual implementation
are not especially profound (though are noteworthy nonetheless), as the main
challenge of this project resided in both the broad and deep prerequisite
understanding required to marry the intricacies of Gillian with the practical
aspects of the DAP, VSCode's extension API, and modern web development --- all
within the limited time span of an M.Eng.\ final project. There are very few
competing projects in this space; those that do exist either have significantly
different goals (Infer), or exist in a prototype stage (VeriFast), proving the
work completed to be particularly groundbreaking.

The work completed opens a number of avenues for further research and
development, some of which are already being considered.

\sacha{
  What about saying that the debugger could be used to interactively write the tactics?
I.e: oh this is failing? It's clearly because *this* needs to be unfolded.
No worry, in the debugger you can clic that button to hot-fix the code, and you don't have to re-run the whole thing, just continue execution.
You can defend that the debugging information is already fed back into execution, according to 4.2.2, so it seems like a natural path.}


\sacha{you can also mention heap visualisation as future work.}
\myparagraph{Education.}
Looking forwards, work has already been planned to add what little more is
needed for Gillian to be used for educational purposes, i.e. Prof.\ Gardner's
course on separation logic, particularly for coursework use; the ability to step
though each command of a symbolic execution could strengthen students'
understanding of separation logic. This will not only serve as proof of
Gillian's usefulness, but also create a far stronger feedback loop, as Gillian
will be opened to a student body --- a far more diverse group than the
collection of researchers that make up the Verified Software group --- leading
to more rapid and better motivated development in Gillian's future.

\myparagraph{Real-world development.}
The advancements made to the debugger in this project bring Gillian ever closer
to the prospect of industry use. Infer has proven that separation logic analysis
is of merit to the software industry, though the feasibility of a more complete
analysis --- requiring developer knowledge of separation logic --- has not yet
been explored. Exploration of real-world Gillian usage would make the most sense
in more security-conscious contexts; this was a large motivating factor in
Gillian's application to a section of Amazon's AWS encryption SDK.\ Where it
stands now, Gillian could easily be the first to test these waters. \sacha{It would not be the first, but it could be "a better" way}

\myparagraph{A Symbolic DAP.}
After some consideration, it was realized that the usefulness of the extensions
made to the Debug Adapter Protocol might not be exclusive to Gillian; there may
be some possibility for the formalization of a \textit{Symbolic} Debug Adapter
Protocol, or SDAP.\@ In fact, initial research found that this is not the first
time such a thing has been considered, though an attempt at formalizing this
concept has yet to be attempted. Some publications to this end are limited in
scope from being masters' theses~\cite{sdap-arxiv, sdap-aurecchia}, \sacha{careful about sdap-arxiv: deductive verification is not symbolic execution.} while others
only achieve support for debugging a single symbolic tool in a single
IDE~\cite{sdap-colombo, sdap-kps} \sacha{that colombo report has some sexy diagrams!} (which, in fairness, has been the end result
of this project as well).
The closest to an existing solution is MagpieBridge (as introduced in
\autoref{sec:background:magpiebridge}), though that project is largely concerned
with static analysis, instead of stepping through a symbolic execution --- more
along the lines of a language server than a debugging one \sacha{magpie bridge is def not debugging. It's exactly a bridge between Java static analysis tools and the LSP (so in other words it's a useless middleman)}.
The appropriateness of the graphical interface developed in this project must
also be considered, as it may be worth including its functionality in the
specification for an SDAP standard.
