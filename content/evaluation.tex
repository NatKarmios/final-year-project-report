%!TEX root = ../main.tex
\chapter{Evaluation plan}
\label{cha:eval}

The measure of success for this project comes down to three factors:

\myparagraph{Evaluating against tools}
As it stands, VeriFast is Gillian's closest competitor; once the project is
completed, Gillian's debugger will be re-compared against VeriFast to see which
has the more usable debugging experience, and why. Another comparison worth
considering is Infer; given its widespread use in industry, the pursuit of
pushing Gillian's ease of use and clarity of error reporting to be comparible
to that of Facebook's tool is a worthy goal.

\myparagraph{Evaluating against examples}
This is a more personal form of evaluation; before embarking on this project,
the author of this project sat Professor Philippa Gardner's course on
separation logic; common challenges set in this course were to create lemmas,
pre-/post-conditions, loop invariants, etc. that are appropriate for particular
programs, and to verify their correctness. A good measure of Gillian's progress
would be to attempty to use it to aid with solving such challenges.

\myparagraph{Pushing the debugger's limits}
The final frontier, as it were - upon completion of this project's
implementation, the debugger ought to be pushed to its limits, exploring the
`corners' of the debugger and what it can accomplish. This will be both a good
test of the debugger's capabilities, and a good indication of what future work
may be needed.
